
\documentclass[11pt]{article}
\usepackage{geometry} % see geometry.pdf on how to lay out the page. There's lots.
\usepackage{amsthm}
\usepackage{hyperref}
\geometry{a4paper} % or letter or a5paper or ... etc
% \geometry{landscape} % rotated page geometry

% See the ``Article customise'' template for come common customisations

\title{Noether's Theorem}
\author{Didi Park}
\date{2 May 2014}
%%% BEGIN DOCUMENT
\begin{document}

\maketitle

%%%%
\section{Preliminaries}
Noether's theorem, in a sentence, states that for every differentiable (infinitesimal) symmetry, there exists a corresponding conservation law. In order to fully understand this, we must understand what symmetry means. We will define symmetry in terms of extremals and invariance. 

%%%%
\section{Functionals}
\subsubsection{Formal statement}
A functional is a mapping from a set of functions to the real numbers, given by

$$J = \int_a^b L(t,q_i,\dot{q}_i)dt.$$

The domain of the mapping is the set of twice differentiable functions ${q(t)}$ on the closed interval $[a,b]$.

The integrand of the functional, $L(t,q_i,\dot{q}_i)$, is called the \emph{Lagrangian} of the functional.

Each $q_i$ represents a generalized coordinate; $i=1,2,\dots,N$.

%%%%
\section{Extremals}
\subsection{Euler-Lagrange Equation}
Given functional
$$J = \int_a^b L(t,q_i,\dot{q}_i)dt,$$
The $\{q_i(t)\}$ that make $J$ an extremal are the N solutions of the Euler-Lagrange equations
$$\frac{\partial L}{\partial q_i}=\frac{d}{dt} \frac{\partial L}{\partial \dot{q_i}}.$$

\subsection{Canonically Conjugate Momentum}
The momentum canonically conjugate to $q_i$ is defined by 
$$p_i \equiv \frac{\partial L}{\partial \dot{q_i}}.$$

This allows us to rewrite the Euler-Lagrange equation as
$$\frac{\partial L}{\partial q_i}=\dot{p_i},$$
which leads to a conservation law!
$p_i$ is conserved when $\frac{\partial L}{\partial q_i}=0.$
Using the chain rule, we take the derivative of the Lagrangian with respect to time:
$$\frac{dL}{dt}=\frac{\partial L}{\partial t}+\frac{\partial L}{\partial q_i}\dot{q_i}+\frac{\partial L}{\partial \dot{q_i}}\ddot{q_i}$$
$$=\frac{\partial L}{\partial t}+\dot{p_i}\dot{q_i}+p_i\ddot{q_i}$$
$$=\frac{d}{dt}[L-p_i\dot{q_i}].$$
This leads to\dots

\subsection{The Hamiltonian}
$$H\equiv H(t,q_i,p_i)\equiv p_i \dot{q_i}-L$$
The Hamiltonian is different from the Lagrangian in that it depends on the canonical momenta.\\
It also has its own symmetry:
$$H \textrm{ is constant  iff } \frac{\partial L}{\partial t}=0.$$ 
It gives us two ways of writing the Euler-Lagrange equation:
$$\frac{\partial L}{\partial t}=-\dot{H}$$
$$\frac{\partial L}{\partial q_i}=\dot{p_i}.$$

%%%%
\section{Invariance}
The functional 
$$J=\int_a^b L(t,q^\mu, \dot{q}^\mu)dt$$
is said to be invariant under the infinitesimal transformation
$$ t'=t+\epsilon \tau+\dots,$$
$$q_i'=q_i+\epsilon \zeta_i+\dots$$
iff
$$J'-J\sim \epsilon^s, \textrm{ where } s>1.$$
\subsection{Rund-Trautman Identity}
If the functional
$$J = \int_a^b L(t,q_i,\dot{q}_i)dt,$$
is invariant under the infinitesimal transformation
$$t'=t+\epsilon \tau+\dots,$$
$$q_i'=q_i+\epsilon\zeta_i+\dots,$$
then
$$\frac{\partial L}{\partial q_i}\zeta_i+p_i\dot{\zeta_i}+\frac{\partial L}{\partial t}\tau - H\dot{\tau}=0.$$
Another equivalent form:
$$-(\zeta_i-\dot{q_i}\tau)[\frac{\partial L}{\partial q_i}-\frac{d}{dt}\frac{\partial L}{\partial \dot{q_i}}]=\frac{d}{dt}[p_i\zeta_i-H\tau].$$
%%%%
\section{Statement of the theorem}
If the functional

$$J=\int_a^b L(t,q_i, \dot{q_i})dt$$
is an extremal, and if under the infinitesimal transformation
$$t'=t+\epsilon\tau+\dots,\\
	q_i'=q_i+\epsilon\zeta_i+\dots
$$
the functional is invariant according to the definition
$$L'\frac{dt'}{dt}-L=\epsilon\frac{dF}{dt}+O(\epsilon^s), \textrm{ where } s>1,$$
then the following conservation law holds:
$$p+\mu\zeta^\mu-H\tau-F=\textrm{const.}$$
%If that's hard to grasp, here's an equivalent aphorism:\\\\
%\emph{For every differentiable symmetry, there exists a corresponding conservation law.}
\subsection{Cute proof}
If J is an extremal, then the Euler-Lagrange equations hold:
$$\frac{\partial L}{\partial t}=-\dot{H}$$
$$\frac{\partial L}{\partial x_i}=\dot{p_i}.$$
If J is also invariant, then the Euler-Lagrange equations, when substituted into the Rund-Trautman identity, make
$$\frac{d}{dt}[p_i\zeta_i-H\tau-F]=0.$$
Therefore,
$$p_i\zeta_i-H\tau-F=\textrm{const.}$$

\subsection{Simple proof}
Suppose that the functional
$$J = \int_{t_1}^{t_2} L(t,q_i,\dot{q}_i)dt.$$
is perturbed by a symmetric, infinitesimal translation $q_i\rightarrow q_i+\epsilon f_i(q).$
$$\delta A = \int_{t_1}^{t_2} \frac{\partial L}{\partial q_i}\delta{q_i}+\frac{\partial L}{\partial \dot{q_i} }\delta{\dot{q_i}}=0$$
Integrating by parts, we get
$$ = \int_{t_1}^{t_2} [\frac{\partial L}{\partial q_i} \delta q_i - \frac{d}{dt} \frac{\partial L}{\partial \dot{q_i}}\delta{q_i}] 
+\left. \frac{\partial L}{\partial \dot{q_i}} \right|_{t_1}^{t_2}  $$
By the Euler-Lagrange equation, the integrand equals 0.
$$=\left. \frac{\partial L}{\partial \dot{q_i}} \right|_{t_1}^{t_2}=0. $$
This can be rewritten as
$$=\frac{\partial L}{\partial q_i}\epsilon f_i(q) = p_i f_i(q)=0.$$
This proof, being simpler, gives a different conserved quantity from the first proof, but equivalently shows that there exists a corresponding conservation law. We call the function $f_i(q)$ a Noether charge. 

\subsection{Conservation of Energy}
We translate a particle through time using the translation
$$q(t)\rightarrow q(t-\epsilon)$$
which implies
$$\delta q(t)=-\frac{dq}{dt}\epsilon$$
$$\delta q = -\dot{q}\epsilon$$
We assume that it is a symmetry, so
$$\delta L = 0$$
$$=\int_{t_1}^{t_2} dq \frac{\partial L}{\partial q}(-\dot{q}\epsilon) + \frac{\partial L}{\partial q}\delta \dot{q} + A-B$$
Integrating by parts, we obtain
$$=\int_{t_1}^{t_2}  [\frac{\partial L}{\partial q}- \frac{d}{dt}\frac{\partial L}{\partial \dot{q}}]\delta q+ \left. \delta q \frac{\partial L}{\partial \dot{q}} \right|_{t_1}^{t_2}+A-B$$
By the Euler-Lagrange equation,
$$= \left. \delta q \frac{\partial L}{\partial \dot{q}} \right|_{t_1}^{t_2}+A-B$$
$$=\left. -\epsilon \dot{q} \frac{\partial L}{\partial \dot{q}} \right|_{t_1}^{t_2}+A-B$$
We know $A=L(t_2)\epsilon$ and $B=L(t_1)\epsilon,$ so 
$$=\left. [L(t) -\epsilon \dot{q} \frac{\partial L}{\partial \dot{q}}] \right|_{t_1}^{t_2}.$$
This is the conserved quantity, so we can simply negate it:
$$=\left. [\epsilon \dot{x} \frac{\partial L}{\partial \dot{q}}-L(t)] \right|_{t_1}^{t_2}$$
which is called the Hamiltonian $H$.
It turns out that this equals energy.
Setting $L(t)=K-U,$
$$=\left. [\epsilon \dot{x}\dot{p}-\frac{1}{2}m\dot{x^2}+U(x)] \right|_{t_1}^{t_2}$$
$$=\left. [m\dot{x^2}-\frac{1}{2}m\dot{x^2}+U(x)] \right|_{t_1}^{t_2}$$
$$=\left. [\frac{1}{2}m\dot{x^2}+U(x)] \right|_{t_1}^{t_2}=0.$$
Yay, energy is conserved!

\subsection{Conservation of Linear Momentum}
We translate a particle using the infinitesimal translation
$$ \delta x = -\epsilon$$
$$ \delta y = 0$$
Assuming that this translation is a symmetry, we calculate the Noether charge:
$$\epsilon f_x=-\epsilon \Rightarrow f_x=-1$$
$$\epsilon f_y = 0 \Rightarrow f_y=0$$
So the Noether charge (conserved quantity) is:
$$p_x\textrm{ (linear momentum).}$$
\subsection{Conservation of Angular Momentum}
We rotate a particle about the origin using the infinitesimal translation
$$ \delta x = -\epsilon y$$
$$ \delta y = \epsilon x.$$
Assuming that this translation is a symmetry, we calculate the Noether charge:
$$\epsilon f_x=-\epsilon y \Rightarrow f_x=-y$$
$$\epsilon f_y = \epsilon x \Rightarrow f_y=x$$
So the Noether charge (conserved quantity) is:
$$-p_xy+p_yx=L\textrm{ (angular momentum).}$$
%%%%
\section{Problems}
\begin{enumerate}

\item Derive Snell's law
$$n_1sin(\theta_1)=n_2sin(\theta_2)$$
 using the functional $T=\frac{1}{c}[n_1s_1+n_2s_2].$\\
Hint: Fermat's principle states that the extremal path T must be a minimum.

\item Consider the damped oscillator with Lagrangian
$$L=[\frac{1}{2}m\dot{x^2}-\frac{1}{2}kx^2]e^{\frac{bt}{m}}.$$
Apply the transformation
$$t'=t+\epsilon\tau, x'=x+\epsilon\zeta\textrm{, where }\tau=1\textrm{ and }\zeta=\frac{-bx}{2m}.$$
What is the corresponding conservation law?

\item Consider the functional with Lagrangian $L=L(t,\dot{x})=t\dot{x^2}.$
\begin{enumerate}
\item Find the canonical momentum and the Hamiltonian.
\item Using $\tau=At$ and $\zeta=x_0$, find the corresponding conservation law.
\end{enumerate}
\end{enumerate}

%%%%
\section{Further reading/Works Cited}
\begin{enumerate}
\item Baez, John. \url{http://math.ucr.edu/home/baez/noether.html}.
\item Neuenschwander, Dwight E. (2010). \emph{Emmy Noether's Wonderful Theorem}.
\item Susskind, Leonard. \url{https://www.youtube.com/watch?v=FZDy_Dccv4s}
\item Thompson, Rob. \url{http://www.math.umn.edu/~robt/noethertalk.pdf}
\item Wikipedia. \url{http://en.wikipedia.org/wiki/Noether%27s_theorem}
\end{enumerate}
\end{document}