
\documentclass[11pt]{article}
\usepackage[]{geometry} % see geometry.pdf on how to lay out the page. There's lots.
\usepackage{amsthm}
\usepackage{hyperref}
\geometry{a4paper} % or letter or a5paper or ... etc
% \geometry{landscape} % rotated page geometry

% See the ``Article customise'' template for come common customisations

\title{Lexemes, Lexicons and Lexicalization}
\author{Didi Park}
\date{21 November 2014}
%%% BEGIN DOCUMENT
\begin{document}

\maketitle

%%%%
\section{Lexemes and Lexicons}
Lexemes are the smallest lexical unit\footnote{similarly, phonemes are the smallest unit of meaningful sound in a language, and morphemes are the smallest units of meaning in a language}; they are independent of inflection. A lexeme can be thought of as the set of all forms of the lemma: ``eat'' can be thought of as a lemma, and ``ate,'' ``eats'', ``eaten,'' ``eater,'' are some of its lexemes. These words all have the same fundamental meaning, regardless of grammatical structure\footnote{the inflections `-s,' `-en,' `-er' are examples of morphemes, as they modify the grammatical meaning and structure of `eat'.}.

In programming languages, the concept of lexemes is relevant: In the Java statement \begin{verbatim}private boolean = true;\end{verbatim} the parser must be able to differentiate between different meaningful units -- lexemes. The statement is thus split up into ``private,'' ``boolean'', ``=,'' and ``true'' by using whitespace as a separator.

\section{Lexicons}
Lexicons are collections of lexemes, and can more simply be thought of as the set of all words of a language. Each language has a different lexicon. Lexicons can be organized by its closed and open categories. Closed categories are rarely, if ever modified. Conjunctions are a good example: ``and,'' ``or,'' ``but,'' etc. are some of the most commonly used and functionally useful lexemes in our language. 

\section{Lexicalization}
Lexicalization is the process by which new words enter the lexicon.
\subsection{}
%%%%
\section{Further reading/Works Cited}
\begin{enumerate}
\item Wikipedia. \url{http://en.wikipedia.org/wiki/Lexical_analysis}
\end{enumerate}
\end{document}