% !TEX TS-program = pdflatex
% !TEX encoding = UTF-8 Unicode

% This is a simple template for a LaTeX document using the "article" class.
% See "book", "report", "letter" for other types of document.

\documentclass[11pt]{article} % use larger type; default would be 10pt

\usepackage[utf8]{inputenc} % set input encoding (not needed with XeLaTeX)

%%% Examples of Article customizations
% These packages are optional, depending whether you want the features they provide.
% See the LaTeX Companion or other references for full information.

%%% PAGE DIMENSIONS
\usepackage{geometry} % to change the page dimensions
\geometry{a4paper} % or letterpaper (US) or a5paper or....
% \geometry{margin=2in} % for example, change the margins to 2 inches all round
% \geometry{landscape} % set up the page for landscape
%   read geometry.pdf for detailed page layout information

\usepackage{graphicx} % support the \includegraphics command and options

% \usepackage[parfill]{parskip} % Activate to begin paragraphs with an empty line rather than an indent

%%% PACKAGES
\usepackage{booktabs} % for much better looking tables
\usepackage{array} % for better arrays (eg matrices) in maths
\usepackage{paralist} % very flexible & customisable lists (eg. enumerate/itemize, etc.)
\usepackage{verbatim} % adds environment for commenting out blocks of text & for better verbatim
\usepackage{subfig} % make it possible to include more than one captioned figure/table in a single float
% These packages are all incorporated in the memoir class to one degree or another...

%%% HEADERS & FOOTERS
\usepackage{fancyhdr} % This should be set AFTER setting up the page geometry
\pagestyle{fancy} % options: empty , plain , fancy
\renewcommand{\headrulewidth}{0pt} % customise the layout...
\lhead{}\chead{}\rhead{}
\lfoot{}\cfoot{\thepage}\rfoot{}

%%% SECTION TITLE APPEARANCE
\usepackage{sectsty}
\allsectionsfont{\sffamily\mdseries\upshape} % (See the fntguide.pdf for font help)
% (This matches ConTeXt defaults)

%%% ToC (table of contents) APPEARANCE
\usepackage[nottoc,notlof,notlot]{tocbibind} % Put the bibliography in the ToC
\usepackage[titles,subfigure]{tocloft} % Alter the style of the Table of Contents
\renewcommand{\cftsecfont}{\rmfamily\mdseries\upshape}
\renewcommand{\cftsecpagefont}{\rmfamily\mdseries\upshape} % No bold!

%%% END Article customizations

%%% The "real" document content comes below...

\title{The Gamma Function}
\author{didigodot}
%\date{} % Activate to display a given date or no date (if empty),
         % otherwise the current date is printed 

\begin{document}
\maketitle

[I'm starting a series of posts on the gamma function and its representations]

First off, what is the gamma function?

Well, it's like the factorial function, only extended to the complex numbers.

Recall that $n!=n(n-1)!^{\dagger}$

Anyway, back to the gamma function. 

We want a function which can compute factorials for all real and complex numbers, not just non-negative integers.

Here's a clever way of writing $x!$:

$$x!=\frac{(n+x)!}{(n+x)(n-1+x)\dots(1+x)}$$

$$=\frac{(n+x)(n+x-1)(n+x-2)\dots(n+1)n!}{(n+x)(n-1+x)\dots(1+x)}$$

In the numerator, we're subtracting from $x$ until $x$ goes away, and in the denominator we're subtracting from $n$ until $n$ goes away.

Now, if we let $n>>x$, then the above expression equals

$$\lim_{n\to\infty}\frac{n^xn!}{(n+x)(n-1+x)\dots(1+x)}$$

And at this point, we've gotten an expression which we can evaluate at non-integer $x$, so we will go ahead and define the gamma function to be

$$\Gamma(z)=(z-1)!^{\ddagger}=\lim_{n\to\infty}\frac{n^{z}n!}{(n+z)(n-1+z)\dots(1+z)z}$$

There we go!

Comments:
$\dagger$ You probably thought of it as $n(n-1)(n-2)\dots$, but this recursive definition is a little more useful because it allows us to figure out what $0!$ is; by the recursive definition, $1!=1\times0!$, which implies that $0!=1$
$\ddagger$ Why shifted to $(z-1)!$ rather than simply $z!$? It's going to make things nicer later on; just trust me.
\end{document}
